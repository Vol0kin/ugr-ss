\documentclass[11pt,a4paper]{article}
\usepackage[spanish,es-nodecimaldot]{babel}	% Utilizar español
\usepackage[utf8]{inputenc}					% Caracteres UTF-8
\usepackage{graphicx}						% Imagenes
\usepackage[hidelinks]{hyperref}			% Poner enlaces sin marcarlos en rojo
\usepackage{fancyhdr}						% Modificar encabezados y pies de pagina
\usepackage{float}							% Insertar figuras
\usepackage[textwidth=390pt]{geometry}		% Anchura de la pagina
\usepackage[nottoc]{tocbibind}				% Referencias (no incluir num pagina indice en Indice)
\usepackage{enumitem}						% Permitir enumerate con distintos simbolos
\usepackage[T1]{fontenc}					% Usar textsc en sections
\usepackage{amsmath}						% Símbolos matemáticos
\usepackage{subcaption}

% Comando para poner el nombre de la asignatura
\newcommand{\asignatura}{Simulación de Sistemas}
\newcommand{\autor}{Vladislav Nikolov Vasilev}
\newcommand{\titulo}{Práctica 4}
\newcommand{\subtitulo}{Modelos de Simulación Dinámicos Contínuos}

% Configuracion de encabezados y pies de pagina
\pagestyle{fancy}
\lhead{\autor{}}
\rhead{\asignatura{}}
\lfoot{Grado en Ingeniería Informática}
\cfoot{}
\rfoot{\thepage}
\renewcommand{\headrulewidth}{0.4pt}		% Linea cabeza de pagina
\renewcommand{\footrulewidth}{0.4pt}		% Linea pie de pagina

\begin{document}
\pagenumbering{gobble}

% Pagina de titulo
\begin{titlepage}

\begin{minipage}{\textwidth}

\centering

\includegraphics[scale=0.5]{img/ugr.png}\\

\textsc{\Large \asignatura{}\\[0.2cm]}
\textsc{GRADO EN INGENIERÍA INFORMÁTICA}\\[1cm]

\noindent\rule[-1ex]{\textwidth}{1pt}\\[1.5ex]
\textsc{{\Huge \titulo\\[0.5ex]}}
\textsc{{\Large \subtitulo\\}}
\noindent\rule[-1ex]{\textwidth}{2pt}\\[3.5ex]

\end{minipage}

\vspace{0.5cm}

\begin{minipage}{\textwidth}

\centering

\textbf{Autor}\\ {\autor{}}\\[2.5ex]
\textbf{Rama}\\ {Computación y Sistemas Inteligentes}\\[2.5ex]
\vspace{0.3cm}

\includegraphics[scale=0.3]{img/etsiit.jpeg}

\vspace{0.7cm}
\textsc{Escuela Técnica Superior de Ingenierías Informática y de Telecomunicación}\\
\vspace{1cm}
\textsc{Curso 2019-2020}
\end{minipage}
\end{titlepage}

\pagenumbering{arabic}
\tableofcontents
\thispagestyle{empty}				% No usar estilo en la pagina de indice

\newpage

\setlength{\parskip}{1em}

\section{Introducción}

El objetivo de esta práctica es el estudio de un modelo dinámico discreto basado
en las ecuaciones de Lotka-Volterra. Estas ecuaciones permiten estudiar ecosistemas
con dos especies relacionadas entre sí: las \textbf{presas} y los \textbf{depredadores}.
Estas ecuaciones se pueden ver a continuación:

\begin{equation}
\begin{split}
\frac{dx}{dt} = a_{11}x - a_{12}xy \\
\frac{dy}{dt} = a_{21}xy - a_{22}y
\end{split}
\end{equation}

\noindent donde $x$ representa la población de presas e $y$ representa la población
de depredadores. El resto de parámetros, en este caso, respresentan lo siguiente:

\begin{itemize}[label=\textbullet]
	\item $a_{11}$ representa la tasa de crecimiento de las presas.
	\item $a_{12}$ representa una la efectividad de los depredadores a la hora de
	capturar las presas.
	\item $a_{21}$ representa el valor nutricional de las presas (a mayor valor, menos
	presas será necesario que cace un depredador para sobrevivir).
	\item $a_{22}$ representa la tasa de crecimiento de los depredadores. En este caso,
	un valor más grande indica que la población decrece más, mientras que un valor más
	bajo indica que decrece en menor medida, y por tanto, los miembros de la población
	son más longevos.
\end{itemize}

Una vez explicado el sistema a estudiar y una vez que se ha implementado siguiendo
las instrucciones proporcionadas, vamos a experimentar con los parámetros del sistema
para ver cómo evolucionan las dos especies en función de estos. También haremos
una comparativa entre los dos métodos de integración que se proponen: el método de Euler
y el de Runge-Kutta.

Para todas las experimentaciones vamos a simular desde $\texttt{tinic} = 0$ hasta $\texttt{tfin} = 200$,
ya que para valores más altos de \texttt{tfin} no se pueden apreciar bien
los resultados en las gráficas. El intervalo de cálculo será en un principio de 0.1, pero a la hora
de comparar los distintos métodos de integración será variable. Los demás valores se van
a ir variando con el objetivo de estudiar el funcionamiento del sistema.

\section{Experimentación con las condiciones iniciales}

Lo primero que vamos a hacer es experimentar con las condiciones iniciales, con el
objetivo de ver cómo cambia el comportamiento del sistema en función de estos.

Lo primero que vamos a hacer es establecer los valores de los parámetros $a_{ij}$,
ya que estos se mantendrán fijos a lo largo de la experimentación. En este caso,
los valores son los siguientes:

$$
	a_{11} = 5, \; a_{12} = 0.05, \; a_{21} = 0.0004, \; a_{22} = 0.2
$$

En un primer instante, se pide que los valores iniciales de las dos poblaciones
sean $x = \frac{a_{22}}{a_{21}}$ e $y = \frac{a_{11}}{a_{12}}$. Si los calculamos,
obtenemos que $x = 500$ e $y = 100$. Si ahora ejecutamos el simulador con los parámetros
correspondientes, obtenemos los siguientes resultados:

\begin{figure}[H]
	\centering
	\includegraphics[scale=0.6]{img/x500y100}
	\caption{Simulación del sistema con 500 presas ($x$) y 100 depredadores ($y$).}
\end{figure}

Vemos que en este caso el sistema permanece en equilibrio, ya el número de individuos
que componen las dos poblaciones no varía a lo largo del tiempo. Esto se debe a que
los valores iniciales son proporcionales a los parámetros $a_{ij}$, de forma que
el número inicial de presas depende de la tasa de crecimiento de los depredadores
y del valor nutricional de las presas, mientras que el número inicial de depredadores
depende de la tasa de incremento de las presas y de la efectividad de los depredadores.
De esta forma, nos podemos asegurar que el número inicial de individuos sea el justo para
que en ningún momento exista una fluctuación en el número de individuos que componen las
poblaciones.

Habiendo visto la situación de equilibrio anterior, vamos a ver ahora qué es lo que pasa
cuando se modifican los valores iniciales para las dos poblaciones. Vamos a probar, por ejemplo,
con un sistema en el que hayan más presas, conservando el mismo número de depredadores.
Por ejemplo, vamos a establecer que $x = 600$, para ver qué es lo que sucede al salir de la
zona de equilibrio. Los resultados se pueden ver a continuación:

\begin{figure}[H]
\centering
\begin{subfigure}{.5\textwidth}
	\centering
	\includegraphics[scale=0.45]{img/x600y100}
	\subcaption{Evolución de las dos poblaciones en conjunto}
	\label{fig:x600}
\end{subfigure}%
\begin{subfigure}{.5\textwidth}
	\centering
	\includegraphics[scale=0.45]{img/circulo-x600}
	\subcaption{Número de presas en función del número de depredadores.}
	\label{fig:circ-x600}
\end{subfigure}
\caption{Resultados para un número inicial de 600 presas.}
\end{figure}

Tal y como vemos en la figura \ref{fig:x600}, al aumentar el número inicial de presas y hacer que
su valor no dependa de los parámetros $a_{ij}$ se empiezan a producir cambios en las dos poblaciones.
Vemos que ambas poblaciones van creciendo y dereciendo de la misma forma periódicamente. Vemos también
que estos crecimientos están acotados, ya que no se superan ciertos rangos de valores. Se puede observar
que cuando la población de presas disminuye, se produce un aumento en la población de depredadores.
Lo mismo sucede en el caso contrario: cuando decrece la de depredadores, aumenta la de presas. Estas evoluciones
son completamente lógicas y razonables, ya que a menos presas, menos comida tendrán los depredadores, y por tanto
un mayor número de ellos morirá. A más presas, más comida, y por tanto más va a crecer la población de depredadores.
La evolución de las dos poblaciones se puede ver también en la figura \ref{fig:circ-x600}, donde vemos
la evolución de la población de presas en función de la de depredadores. Vemos que la evolución tiene
forma circular, casi como una elipse. Por tanto, observando ambas gráficas, podemos concluir que con estos
valores iniciales, las dos poblaciones crecen y decrecen de forma equilibrada, sin extinguirse ninguna de ellas
en ningún momento.

Ahora, si probamos a modificar el número inicial de depredadores y hacemos que tenga el valor
$y = 150$, sin modificar el número de presas iniciales (es decir, dejando dicho número a 500),
obtenemos los siguientes resultados:

\begin{figure}[H]
\centering
\begin{subfigure}{.5\textwidth}
	\centering
	\includegraphics[scale=0.45]{img/x500y150}
	\subcaption{Evolución de las dos poblaciones en conjunto}
	\label{fig:y150}
\end{subfigure}%
\begin{subfigure}{.5\textwidth}
	\centering
	\includegraphics[scale=0.45]{img/circulo-y150}
	\subcaption{Número de presas en función del número de depredadores.}
	\label{fig:circ-y150}
\end{subfigure}
\caption{Resultados para un número inicial de 150 depredadores.}
\end{figure}

Vemos que en este caso sucede algo parecido al anterior, ya que no se extingue ninguna de las especies.
No obstante, la evolución de las dos poblaciones es algo diferente. Aquí podemos ver que los aumentos
y las disminuciones del número de presas es mucho más prounciado que en el caso anterior, ya que los
valores máximos alcanzados son mucho más altos que en el caso anterior. Los mínimos también son muy diferentes,
ya que se quedan bastante cerca de 0, es decir, al borde de la extinción. El crecimiento de la población
de depredadores también ha experimentado un ligero cambio, ya que ahora se alcanzan valores máximos algo
más altos. Además, la forma del crecimiento ha cambiado, ya que ha pasado de tener una forma sinusoidal a otra
que recuerda más a una ``ola''. Si ahora observamos la figura \ref{fig:circ-y150} vemos que, efectivamente, ha
habido un cambio bastante importante respecto al caso anterior. La forma ha dejado de parecerse a una elipse para
parecer una circunferencia bastante deformada. La parte de abajo está bastante aplanada, lo cuál implica
un decrecimiento muy rápido del número de depredadores una vez que el de presas es casi 0. Después, se produce
un crecimiento en el número de presas hasta llegar al máximo, y posteriormente un decrecimiento de nuevo debido
a que el número de depredadores se ha visto incrementado debido a la presencia de más comida. Por tanto, a la vista
de los resultados, vemos que en este sistema también hay una evolución equilibrada de las dos especies, sin que
ninguna de ellas se extinga.

\section{Experimentación con los parámetros}

Una vez que hemos experimentado con los valores iniciales de las dos poblaciones, vamos
a probar a modificar los parámetros $a_{ij}$ para ver qué cambios se producen en el sistema.
Para ello, vamos a partir del sistema con 600 presas y 100 depredadores, ya que es el que tiene
una evolución más balanceada de los dos que hemos probado anteriormente.

\section{Comparativa de los métodos de integración}

\end{document}

