\documentclass[11pt,a4paper]{article}
\usepackage[spanish,es-nodecimaldot]{babel}	% Utilizar español
\usepackage[utf8]{inputenc}					% Caracteres UTF-8
\usepackage{graphicx}						% Imagenes
\usepackage[hidelinks]{hyperref}			% Poner enlaces sin marcarlos en rojo
\usepackage{fancyhdr}						% Modificar encabezados y pies de pagina
\usepackage{float}							% Insertar figuras
\usepackage[textwidth=390pt]{geometry}		% Anchura de la pagina
\usepackage[nottoc]{tocbibind}				% Referencias (no incluir num pagina indice en Indice)
\usepackage{enumitem}						% Permitir enumerate con distintos simbolos
\usepackage[T1]{fontenc}					% Usar textsc en sections
\usepackage{amsmath}						% Símbolos matemáticos

% Comando para poner el nombre de la asignatura
\newcommand{\asignatura}{Simulación de Sistemas}
\newcommand{\autor}{Vladislav Nikolov Vasilev}
\newcommand{\titulo}{Práctica 4}
\newcommand{\subtitulo}{Modelos de Simulación Dinámicos Contínuos}

% Configuracion de encabezados y pies de pagina
\pagestyle{fancy}
\lhead{\autor{}}
\rhead{\asignatura{}}
\lfoot{Grado en Ingeniería Informática}
\cfoot{}
\rfoot{\thepage}
\renewcommand{\headrulewidth}{0.4pt}		% Linea cabeza de pagina
\renewcommand{\footrulewidth}{0.4pt}		% Linea pie de pagina

\begin{document}
\pagenumbering{gobble}

% Pagina de titulo
\begin{titlepage}

\begin{minipage}{\textwidth}

\centering

\includegraphics[scale=0.5]{img/ugr.png}\\

\textsc{\Large \asignatura{}\\[0.2cm]}
\textsc{GRADO EN INGENIERÍA INFORMÁTICA}\\[1cm]

\noindent\rule[-1ex]{\textwidth}{1pt}\\[1.5ex]
\textsc{{\Huge \titulo\\[0.5ex]}}
\textsc{{\Large \subtitulo\\}}
\noindent\rule[-1ex]{\textwidth}{2pt}\\[3.5ex]

\end{minipage}

\vspace{0.5cm}

\begin{minipage}{\textwidth}

\centering

\textbf{Autor}\\ {\autor{}}\\[2.5ex]
\textbf{Rama}\\ {Computación y Sistemas Inteligentes}\\[2.5ex]
\vspace{0.3cm}

\includegraphics[scale=0.3]{img/etsiit.jpeg}

\vspace{0.7cm}
\textsc{Escuela Técnica Superior de Ingenierías Informática y de Telecomunicación}\\
\vspace{1cm}
\textsc{Curso 2019-2020}
\end{minipage}
\end{titlepage}

\pagenumbering{arabic}
\tableofcontents
\thispagestyle{empty}				% No usar estilo en la pagina de indice

\newpage

\setlength{\parskip}{1em}

\section{Introducción}

El objetivo de esta práctica es el estudio de un modelo dinámico discreto basado
en las ecuaciones de Lotka-Volterra. Estas ecuaciones permiten estudiar ecosistemas
con dos especies relacionadas entre sí: las \textbf{presas} y los \textbf{depredadores}.
Estas ecuaciones se pueden ver a continuación:

\begin{equation}
\begin{split}
\frac{dx}{dt} = a_{11}x - a_{12}xy \\
\frac{dy}{dt} = a_{21}xy - a_{22}y
\end{split}
\end{equation}

\noindent donde $x$ representa la población de presas e $y$ representa la población
de depredadores. El resto de parámetros, en este caso, respresentan lo siguiente:

\begin{itemize}[label=\textbullet]
	\item $a_{11}$ representa la tasa de crecimiento de las presas.
	\item $a_{12}$ representa una la efectividad de los depredadores a la hora de
	capturar las presas.
	\item $a_{21}$ representa el valor nutricional de las presas (a mayor valor, menos
	presas será necesario que cace un depredador para sobrevivir).
	\item $a_{22}$ representa la tasa de crecimiento de los depredadores. En este caso,
	un valor más grande indica que la población decrece más, mientras que un valor más
	bajo indica que decrece en menor medida, y por tanto, los miembros de la población
	son más longevos.
\end{itemize}

Una vez explicado el sistema a estudiar y una vez que se ha implementado siguiendo
las instrucciones proporcionadas, vamos a experimentar con los parámetros del sistema
para ver cómo evolucionan las dos especies en función de estos. También haremos
una comparativa entre los dos métodos de integración que se proponen: el método de Euler
y el de Runge-Kutta.

Para todas las experimentaciones vamos a simular desde $\texttt{tinic} = 0$ hasta $\texttt{tfin} = 200$,
ya que para valores más altos de \texttt{tfin} no se pueden apreciar bien
los resultados en las gráficas. El intervalo de cálculo será en un principio de 0.1, pero a la hora
de comparar los distintos métodos de integración será variable. Los demás valores se van
a ir variando con el objetivo de estudiar el funcionamiento del sistema.

\section{Experimentación con las condiciones iniciales}



\section{Experimentación con los parámetros}

\section{Comparativa de los métodos de integración}

\end{document}

