\documentclass[11pt,a4paper]{report}
\usepackage[spanish,es-nodecimaldot]{babel}	% Utilizar español
\usepackage[utf8]{inputenc}					% Caracteres UTF-8
\usepackage{graphicx}						% Imagenes
\usepackage[hidelinks]{hyperref}			% Poner enlaces sin marcarlos en rojo
\usepackage{fancyhdr}						% Modificar encabezados y pies de pagina
\usepackage{float}							% Insertar figuras
\usepackage[textwidth=390pt]{geometry}		% Anchura de la pagina
\usepackage[nottoc]{tocbibind}				% Referencias (no incluir num pagina indice en Indice)
\usepackage{enumitem}						% Permitir enumerate con distintos simbolos
\usepackage[T1]{fontenc}					% Usar textsc en sections
\usepackage{amsmath}						% Símbolos matemáticos

% Comando para poner el nombre de la asignatura
\newcommand{\asignatura}{Simulación de Sistemas}
\newcommand{\autor}{Vladislav Nikolov Vasilev}
\newcommand{\titulo}{Práctica 2}
\newcommand{\subtitulo}{Modelos de Monte Carlo. Generadores de datos}

% Configuracion de encabezados y pies de pagina
\pagestyle{fancy}
\lhead{\autor{}}
\rhead{\asignatura{}}
\lfoot{Grado en Ingeniería Informática}
\cfoot{}
\rfoot{\thepage}
\renewcommand{\headrulewidth}{0.4pt}		% Linea cabeza de pagina
\renewcommand{\footrulewidth}{0.4pt}		% Linea pie de pagina

\begin{document}
\pagenumbering{gobble}

% Pagina de titulo
\begin{titlepage}

\begin{minipage}{\textwidth}

\centering

\includegraphics[scale=0.5]{img/ugr.png}\\

\textsc{\Large \asignatura{}\\[0.2cm]}
\textsc{GRADO EN INGENIERÍA INFORMÁTICA}\\[1cm]

\noindent\rule[-1ex]{\textwidth}{1pt}\\[1.5ex]
\textsc{{\Huge \titulo\\[0.5ex]}}
\textsc{{\Large \subtitulo\\}}
\noindent\rule[-1ex]{\textwidth}{2pt}\\[3.5ex]

\end{minipage}

\vspace{0.5cm}

\begin{minipage}{\textwidth}

\centering

\textbf{Autor}\\ {\autor{}}\\[2.5ex]
\textbf{Rama}\\ {Computación y Sistemas Inteligentes}\\[2.5ex]
\vspace{0.3cm}

\includegraphics[scale=0.3]{img/etsiit.jpeg}

\vspace{0.7cm}
\textsc{Escuela Técnica Superior de Ingenierías Informática y de Telecomunicación}\\
\vspace{1cm}
\textsc{Curso 2019-2020}
\end{minipage}
\end{titlepage}

\pagenumbering{arabic}
\tableofcontents
\thispagestyle{empty}				% No usar estilo en la pagina de indice

\newpage

\setlength{\parskip}{1em}

\chapter{\textsc{Mi Segundo Modelo de Simulación de Monte Carlo}}

\section{Modelización por Monte Carlo}

Una vez que hemos creado nuestro modelo de Monte Carlo, vamos a ver qué resultados obtenemos, y si éstos son buenos o pueden mejorar.
Para poder contrastar, solo podremos utilizar una de las expresiones que encontramos en el guión proporcionado. La correctitud
del resto será confirmada viendo si los resultados obtenidos son parecidos o no a medida que se van aumentando el número de
simulaciones.

Para la experimentación, vamos a probar con distintos valores de $x$, $y$ y número de simulaciones.
Vamos a probar con las siguientes combinaciones de ganancias y pérdidas:

\begin{itemize}
	\item Con $x = 10$ e $y = 1$.
	\item Con $x = 10$ e $y = 5$.
	\item Con $x = 10$ e $y = 10$.
	\item Con $x = 15$ e $y = 10$.
\end{itemize}

Cada combinación se va a simular 100, 1000, 10000 y 100000 veces para ver cómo van evolucionando los resultados, si
éstos se van estabilizando o si van variando mucho. También se va a probar para cada generador, para ver los resultados
que ofrecen.

\begin{table}[H]
\resizebox{\columnwidth}{!}{%
\begin{tabular}{c|c|c|c|c|c}
\textbf{\begin{tabular}[c]{@{}c@{}}Ganancia por\\ unidad vendida\\ (x)\end{tabular}} & \textbf{\begin{tabular}[c]{@{}c@{}}Pérdida por\\ unidad no vendida\\ (y)\end{tabular}} & \textbf{\begin{tabular}[c]{@{}c@{}}Número de\\ repeticiones\end{tabular}} & \textbf{\begin{tabular}[c]{@{}c@{}}Mejor número\\ de unidades\\ pedidas (s)\end{tabular}} & \textbf{\begin{tabular}[c]{@{}c@{}}Mejor\\ ganancia\\ media\end{tabular}} & \textbf{Tiempo (seg)} \\ \hline
10                                                                                   & 1                                                                                      & 100                                                                       & 85                                                                                        & 488.76                                                                    & 0.007848              \\
10                                                                                   & 1                                                                                      & 1000                                                                      & 96                                                                                        & 460.116                                                                   & 0.041666              \\
10                                                                                   & 1                                                                                      & 10000                                                                     & 89                                                                                        & 453.7774                                                                  & 0.161990              \\
10                                                                                   & 1                                                                                      & 100000                                                                    & 86                                                                                        & 451.0508                                                                  & 1.363344              \\ \hline
10                                                                                   & 5                                                                                      & 100                                                                       & 79                                                                                        & 391.15                                                                    & 0.004215              \\
10                                                                                   & 5                                                                                      & 1000                                                                      & 67                                                                                        & 345.19                                                                    & 0.016622              \\
10                                                                                   & 5                                                                                      & 10000                                                                     & 71                                                                                        & 331.6865                                                                  & 0.170900              \\
10                                                                                   & 5                                                                                      & 100000                                                                    & 67                                                                                        & 329.49985                                                                 & 1.358017              \\ \hline
10                                                                                   & 10                                                                                     & 100                                                                       & 48                                                                                        & 309                                                                       & 0.002749              \\
10                                                                                   & 10                                                                                     & 1000                                                                      & 57                                                                                        & 270.66                                                                    & 0.016464              \\
10                                                                                   & 10                                                                                     & 10000                                                                     & 50                                                                                        & 253.072                                                                   & 0.136419              \\
10                                                                                   & 10                                                                                     & 100000                                                                    & 48                                                                                        & 246.02                                                                    & 1.289375              \\ \hline
15                                                                                   & 10                                                                                     & 100                                                                       & 72                                                                                        & 528                                                                       & 0.004562              \\
15                                                                                   & 10                                                                                     & 1000                                                                      & 59                                                                                        & 461.9                                                                     & 0.016942              \\
15                                                                                   & 10                                                                                     & 10000                                                                     & 58                                                                                        & 456.05                                                                    & 0.137328              \\
15                                                                                   & 10                                                                                     & 100000                                                                    & 60                                                                                        & 444.4585                                                                  & 1.320523             
\end{tabular}
}%
\caption{Resultados obtenidos por el modelo utilizando el generador de distribución uniforme.}
\label{tabla1}
\end{table}

Para contrastar los datos podemos utilizar, tal y como hemos dicho antes, la expresión analítica que aparece en el guión:

\begin{itemize}
	\item Para el caso de $x = 10$ e $y = 1$, obtenemos que $s^* = 90$.
	\item Para el caso de $x = 10$ e $y = 5$, obtenemos que $s^* = 66$.
	\item Para el caso de $x = 10$ e $y = 10$, obtenemos que $s^* = 49$.
	\item Para el caso de $x = 15$ e $y = 10$, obtenemos que $s^* = 59$. 
\end{itemize}

De los resultados obtenidos, podemos observar que los valores obtenidos, a medida que se van incrementando el número de
repeticiones, se van acercando más y más a los valores óptimos reales. Esto es normal, ya que, aunque haya números aleatorios
envueltos en el proceso, al hacer muchas repeticiones, de media, el resultado se aproximará al valor óptimo, siempre con
un cierto margen de error. Con pocas repeticiones esto es difícil que pase, ya que no se da mucho margen para
obtener un promedio decente, ya que la aleatoriedad puede hacer que salgan muchos valores en los extremos. Por tanto,
intentar extraer conclusiones con un número tan pequeño de repeticiones sería contraproducente y no reflejaría muy bien
la realidad.

En general, podemos afirmar que los resultados obtenidos son bastante precisos, siempre y cuando hagamos un número
razonable de repeticiones. Por tanto, el modelo parece estar funcionando bien.

A la vista de lo que hemos obtenido, podemos decir que, si mantenemos el valor de $x$ constante y subimos el de $y$,
el número de unidades pedidas media va disminuyendo. Esto puede deberse a que no es viable tener un mayor número si
las pérdidas son más grandes.

Una vez vistos los resultados para el modelo que usa la distribución uniforme, vamos a ver qué obtenemos para el resto.
Vamos a seguir con la distribución proporcional. Los resultados obtenidos se pueden ver en la siguiente tabla:

% Please add the following required packages to your document preamble:
% \usepackage{graphicx}
\begin{table}[H]
\resizebox{\textwidth}{!}{%
\begin{tabular}{c|c|c|c|c|c}
\textbf{\begin{tabular}[c]{@{}c@{}}Ganancia por\\ unidad vendida\\ (x)\end{tabular}} & \textbf{\begin{tabular}[c]{@{}c@{}}Pérdida por\\ unidad no vendida\\ (y)\end{tabular}} & \textbf{\begin{tabular}[c]{@{}c@{}}Número de\\ repeticiones\end{tabular}} & \textbf{\begin{tabular}[c]{@{}c@{}}Mejor número\\ de unidades\\ pedidas (s)\end{tabular}} & \textbf{\begin{tabular}[c]{@{}c@{}}Mejor\\ ganancia\\ media\end{tabular}} & \textbf{Tiempo (seg)} \\ \hline
10                                                                                   & 1                                                                                      & 100                                                                       & 63                                                                                        & 337.51                                                                    & 0.001631              \\
10                                                                                   & 1                                                                                      & 1000                                                                      & 76                                                                                        & 298.187                                                                   & 0.034052              \\
10                                                                                   & 1                                                                                      & 10000                                                                     & 71                                                                                        & 287.7859                                                                  & 0.129329              \\
10                                                                                   & 1                                                                                      & 100000                                                                    & 73                                                                                        & 284.1546                                                                  & 1.019929              \\ \hline
10                                                                                   & 5                                                                                      & 100                                                                       & 56                                                                                        & 218                                                                       & 0.005015              \\
10                                                                                   & 5                                                                                      & 1000                                                                      & 42                                                                                        & 197.925                                                                   & 0.014599              \\
10                                                                                   & 5                                                                                      & 10000                                                                     & 39                                                                                        & 190.266                                                                   & 0.105564              \\
10                                                                                   & 5                                                                                      & 100000                                                                    & 40                                                                                        & 188.95885                                                                 & 1.065668              \\ \hline
10                                                                                   & 10                                                                                     & 100                                                                       & 40                                                                                        & 169.6                                                                     & 0.005048              \\
10                                                                                   & 10                                                                                     & 1000                                                                      & 31                                                                                        & 140.26                                                                    & 0.025562              \\
10                                                                                   & 10                                                                                     & 10000                                                                     & 29                                                                                        & 137.962                                                                   & 0.130958              \\
10                                                                                   & 10                                                                                     & 100000                                                                    & 30                                                                                        & 134.8548                                                                  & 1.031648              \\ \hline
15                                                                                   & 10                                                                                     & 100                                                                       & 44                                                                                        & 329.75                                                                    & 0.004820              \\
15                                                                                   & 10                                                                                     & 1000                                                                      & 31                                                                                        & 255.775                                                                   & 0.035382              \\
15                                                                                   & 10                                                                                     & 10000                                                                     & 35                                                                                        & 252.32                                                                    & 0.104106              \\
15                                                                                   & 10                                                                                     & 100000                                                                    & 37                                                                                        & 250.91575                                                                 & 1.028880             
\end{tabular}%
}
\caption{Resultados obtenidos por el modelo utilizando el generador de distribución proporcional.}
\label{tabla2}
\end{table}

Como no tenemos una expresión analítica con la que comparar, vamos a fijarnos en cómo van evolucionando los valores
de $s$.

Tal y como pasaba en el ejemplo anterior, a medida que vamos aumentando el número de repeticiones, más se aproximan los
valores obtenidos a los óptimos. Podemos ver que con unas pocas repeticiones (unas 100) los resultados se quedan bastante lejos
de los que se obtienen con un mayor número de repeticiones, tal y como pasaba antes. Por tanto, si quisiéramos
sacar unas conclusiones sólidas, tendríamos que fijarnos en los resultados con un número grande de repeticiones, como
por ejemplo 10000 o 100000. Podemos decir que el número óptimo estará próximo a los valores reflejados en la tabla
para ese número de repeticiones.

Los resultados son obviamente distintos a los que podemos observar en la tabla \ref{tabla1}. Esto es así porque
la distribución utilizada es diferente. El efecto que causa usar una distribución diferente a la anterior es que los
valores medios de $s$ se ven reducidos. Esto se debe principalmente a la forma que tiene la distribución proporcional,
ya que es decreciente, y por tanto, las demandas más pequeñas tienen una mayor probabilidad que las grandes. Sin embargo,
las dos tablas tienen dos cosas en común. La primera son los tiempos, ya que no hay mucha diferencia notable. Este resultado no
debe sorprender a nadie, ya que las tablas se han generado antes, y lo único que se está haciendo es recuperar los valores.
Y la segunda es la relación entre el valor de $x$ y el de $y$, la cuál ya se comentó anteriormente.

Visto este generador, vamos a ver qué resultados nos permite obtener el último, el de la distribución ``triangular''.
A continuación se pueden ver los resultados:

% Please add the following required packages to your document preamble:
% \usepackage{graphicx}
\begin{table}[H]
\resizebox{\textwidth}{!}{%
\begin{tabular}{c|c|c|c|c|c}
\textbf{\begin{tabular}[c]{@{}c@{}}Ganancia por\\ unidad vendida\\ (x)\end{tabular}} & \textbf{\begin{tabular}[c]{@{}c@{}}Pérdida por\\ unidad no vendida\\ (y)\end{tabular}} & \textbf{\begin{tabular}[c]{@{}c@{}}Número de\\ repeticiones\end{tabular}} & \textbf{\begin{tabular}[c]{@{}c@{}}Mejor número\\ de unidades\\ pedidas (s)\end{tabular}} & \textbf{\begin{tabular}[c]{@{}c@{}}Mejor\\ ganancia\\ media\end{tabular}} & \textbf{Tiempo (seg)} \\ \hline
10                                                                                   & 1                                                                                      & 100                                                                       & 77                                                                                        & 505.45                                                                    & 0.004564              \\
10                                                                                   & 1                                                                                      & 1000                                                                      & 84                                                                                        & 472.743                                                                   & 0.041040              \\
10                                                                                   & 1                                                                                      & 10000                                                                     & 82                                                                                        & 466.5304                                                                  & 0.141798              \\
10                                                                                   & 1                                                                                      & 100000                                                                    & 82                                                                                        & 464.88623                                                                 & 1.392658              \\ \hline
10                                                                                   & 5                                                                                      & 100                                                                       & 64                                                                                        & 420.25                                                                    & 0.006475              \\
10                                                                                   & 5                                                                                      & 1000                                                                      & 55                                                                                        & 393.64                                                                    & 0.017583              \\
10                                                                                   & 5                                                                                      & 10000                                                                     & 65                                                                                        & 387.3845                                                                  & 0.167988              \\
10                                                                                   & 5                                                                                      & 100000                                                                    & 61                                                                                        & 387.02035                                                                 & 1.394215              \\ \hline
10                                                                                   & 10                                                                                     & 100                                                                       & 49                                                                                        & 366.6                                                                     & 0.002019              \\
10                                                                                   & 10                                                                                     & 1000                                                                      & 51                                                                                        & 351.26                                                                    & 0.043020              \\
10                                                                                   & 10                                                                                     & 10000                                                                     & 52                                                                                        & 336.416                                                                   & 0.142922              \\
10                                                                                   & 10                                                                                     & 100000                                                                    & 51                                                                                        & 334.3226                                                                  & 1.428285              \\ \hline
15                                                                                   & 10                                                                                     & 100                                                                       & 61                                                                                        & 627                                                                       & 0.002402              \\
15                                                                                   & 10                                                                                     & 1000                                                                      & 56                                                                                        & 565.075                                                                   & 0.042256              \\
15                                                                                   & 10                                                                                     & 10000                                                                     & 55                                                                                        & 550.065                                                                   & 0.149202              \\
15                                                                                   & 10                                                                                     & 100000                                                                    & 55                                                                                        & 549.21975                                                                 & 1.365606             
\end{tabular}%
}
\caption{Resultados obtenidos por el modelo utilizando el generador de distribución ``triangular''.}
\label{tabla3}
\end{table}

De nuevo, para extraer unas conclusiones sólidas, vamos a fijarnos en los valores medios obtenidos con un número de repeticiones
más alto, por los motivos comentados anteriormente.

En general, podemos ver que los valores de $s$ obtenidos son más ``céntricos'', debido a la forma de la distribución, ya que
esta tiene más forma de triángulo, y por tanto, los valores más probables estarán en el centro. De nuevo, tal y como pudimos
observar en las tablas \ref{tabla1} y \ref{tabla2}, podemos ver claramente la relación entre los valores de $x$ y de $y$.
Y de nuevo, tal y como pasaba antes, no hay mucha diferencia en los tiempos de ejecución.

Observando los valores de los resultados, podemos ver que en este caso son mucho más próximos que en los anteriores. Incluso
con muy pocas repeticiones, los valores de $s$ obtenidos no distan tanto de aquellos obtenidos con un mayor número de repeticiones,
cosa que sí que sucedía, sobre todo en la tabla \ref{tabla1}.

Por tanto, para concluir este apartado, podemos decir que es importante ejecutar un modelo de Monte Carlo repitiendo
la simulación muchas veces, para así obtener unos resultados más fiables. Los valores obtenidos para cada generador han
sido diferentes, como era de esperar, ya que las distribuciones han sido diferentes. Por tanto, esto nos indica que, a la
hora de diseñar un buen modelo, debemos tener cierta información de como son las distribuciones reales, para así poder
obtener unos resultados más representativos. Sin embargo, el hecho de poder probar distintas distribuciones para ver como
varían los resultados ofrece a los modelos de Monte Carlo mucha flexibilidad y potencia, ya que pueden ser adaptados a
las necesidades específicas del usuario.

\section{Modificaciones del Modelo}

En este apartado se han pedido hacer dos modificaciones al modelo anterior. En el primer caso, no hay pérdida por unidad no
vendida, pero existe una cantidad de dinero $z$ que se debe pagar para realizar una devolución de todo lo que no se
ha vendido. En el segundo caso, suponemos que esa cantidad es relativamente grande, y que a lo mejor interesa asumir las pérdidas
de las unidades no vendidas antes que realizar la devolución.

Vamos a construir las tablas y a ver qué resultados obtenemos para cada situación. Para obtener mejor información, vamos a
evitar hacer los experimentos con pocas simulaciones.

\subsection{Primera modificación}

Para esta modificación, vamos a probar con los siguientes valores:

\begin{itemize}
	\item Con $x = 10$ y $z = 1$.
	\item Con $x = 10$ y $z = 5$.
	\item Con $x = 10$ y $z = 10$.
	\item Con $x = 10$ y $z = 100$.
\end{itemize}

Vamos a ver qué resultados obtenemos para cada generador. Por tanto, vamos a comenzar con el generador que utiliza
una distribución uniforme. A continuación se pueden ver los resultados obtenidos:

% Please add the following required packages to your document preamble:
% \usepackage{graphicx}
\begin{table}[H]
\resizebox{\textwidth}{!}{%
\begin{tabular}{c|c|c|c|c|c|c}
\textbf{\begin{tabular}[c]{@{}c@{}}Ganancia por\\ unidad vendida\\ (x)\end{tabular}} & \textbf{\begin{tabular}[c]{@{}c@{}}Pérdida por\\ unidad no vendida\\ (y)\end{tabular}} & \textbf{\begin{tabular}[c]{@{}c@{}}Precio\\ devolución\\ (z)\end{tabular}} & \textbf{\begin{tabular}[c]{@{}c@{}}Número de\\ repeticiones\end{tabular}} & \textbf{\begin{tabular}[c]{@{}c@{}}Mejor número\\ de unidades\\ pedidas (s)\end{tabular}} & \textbf{\begin{tabular}[c]{@{}c@{}}Mejor\\ ganancia\\ media\end{tabular}} & \textbf{Tiempo (seg)} \\ \hline
10                                                                                   & 0                                                                                      & 1                                                                          & 10000                                                                     & 96                                                                                        & 495.5817                                                                  & 0.165755              \\
10                                                                                   & 0                                                                                      & 1                                                                          & 100000                                                                    & 96                                                                                        & 494.01322                                                                 & 1.318989              \\ \hline
10                                                                                   & 0                                                                                      & 5                                                                          & 10000                                                                     & 92                                                                                        & 491.792                                                                   & 0.139747              \\
10                                                                                   & 0                                                                                      & 5                                                                          & 100000                                                                    & 98                                                                                        & 491.2576                                                                  & 1.345592              \\ \hline
10                                                                                   & 0                                                                                      & 10                                                                         & 10000                                                                     & 95                                                                                        & 487.091                                                                   & 0.162312              \\
10                                                                                   & 0                                                                                      & 10                                                                         & 100000                                                                    & 96                                                                                        & 485.581                                                                   & 1.364294              \\ \hline
10                                                                                   & 0                                                                                      & 100                                                                        & 10000                                                                     & 89                                                                                        & 403.8810                                                                  & 0.168761              \\
10                                                                                   & 0                                                                                      & 100                                                                        & 100000                                                                    & 86                                                                                        & 401.7571                                                                  & 1.358222             
\end{tabular}%
}
\caption{Resultados obtenidos por el modelo utilizando el generador de distribución uniforme.}
\label{tabla4}
\end{table}

\subsection{Segunda modificación}

\newpage

\chapter{\textsc{Generadores de Datos}}

\section{Mejorando los Generadores}

\subsection{Reordenación de los valores}

\subsection{Mejorando la búsqueda: búsqueda binaria}

\subsection{Tiempo de acceso constante a la tabla}

\section{Generadores congruenciales lineales}

\newpage

\begin{thebibliography}{5}

\bibitem{nombre-referencia}
Texto referencia
\\\url{https://url.referencia.com}

\end{thebibliography}

\end{document}

