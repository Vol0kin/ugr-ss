\documentclass[11pt,a4paper]{article}
\usepackage[spanish,es-nodecimaldot]{babel}	% Utilizar español
\usepackage[utf8]{inputenc}					% Caracteres UTF-8
\usepackage{graphicx}						% Imagenes
\usepackage[hidelinks]{hyperref}			% Poner enlaces sin marcarlos en rojo
\usepackage{fancyhdr}						% Modificar encabezados y pies de pagina
\usepackage{float}							% Insertar figuras
\usepackage[textwidth=390pt]{geometry}		% Anchura de la pagina
\usepackage[nottoc]{tocbibind}				% Referencias (no incluir num pagina indice en Indice)
\usepackage{enumitem}						% Permitir enumerate con distintos simbolos
\usepackage[T1]{fontenc}					% Usar textsc en sections
\usepackage{amsmath}						% Símbolos matemáticos

% Comando para poner el nombre de la asignatura
\newcommand{\asignatura}{Simulación de Sistemas}
\newcommand{\autor}{Vladislav Nikolov Vasilev}
\newcommand{\titulo}{Práctica 3}
\newcommand{\subtitulo}{Modelos de Simulación Dinámicos y Discretos}

% Configuracion de encabezados y pies de pagina
\pagestyle{fancy}
\lhead{\autor{}}
\rhead{\asignatura{}}
\lfoot{Grado en Ingeniería Informática}
\cfoot{}
\rfoot{\thepage}
\renewcommand{\headrulewidth}{0.4pt}		% Linea cabeza de pagina
\renewcommand{\footrulewidth}{0.4pt}		% Linea pie de pagina

\begin{document}
\pagenumbering{gobble}

% Pagina de titulo
\begin{titlepage}

\begin{minipage}{\textwidth}

\centering

\includegraphics[scale=0.5]{img/ugr.png}\\

\textsc{\Large \asignatura{}\\[0.2cm]}
\textsc{GRADO EN INGENIERÍA INFORMÁTICA}\\[1cm]

\noindent\rule[-1ex]{\textwidth}{1pt}\\[1.5ex]
\textsc{{\Huge \titulo\\[0.5ex]}}
\textsc{{\Large \subtitulo\\}}
\noindent\rule[-1ex]{\textwidth}{2pt}\\[3.5ex]

\end{minipage}

\vspace{0.5cm}

\begin{minipage}{\textwidth}

\centering

\textbf{Autor}\\ {\autor{}}\\[2.5ex]
\textbf{Rama}\\ {Computación y Sistemas Inteligentes}\\[2.5ex]
\vspace{0.3cm}

\includegraphics[scale=0.3]{img/etsiit.jpeg}

\vspace{0.7cm}
\textsc{Escuela Técnica Superior de Ingenierías Informática y de Telecomunicación}\\
\vspace{1cm}
\textsc{Curso 2019-2020}
\end{minipage}
\end{titlepage}

\pagenumbering{arabic}
\tableofcontents
\thispagestyle{empty}				% No usar estilo en la pagina de indice

\newpage

\setlength{\parskip}{1em}

\section{\textsc{Mi segundo modelo de simulación discreto}}

En esta sección vamos a estudiar primero el comportamiento de un modelo de
simulación de un servidor con una única cola, y después de $m$ servidores
con una única cola. Vamos a ver cómo distintos métodos de incremento del
itempo pueden afectar al funcionamiento del sistema, y discutiremos cuál
de ellos es mejor.

\subsection{Método de incremento fijo del tiempo}

El primer método de incremento del tiempo que vamos a estudiar es el incremento
fijo. Como su propio nombre indica, el tiempo se va incrementando en una cantidad
fija, tal como lo hace un reloj normal. Esta cantidad viene decidida por la persona
que va a utilizar el sistema (pueden ser minutos, segundos, milésimas, horas, etc.).

Debido a la naturaleza de dicho incremento, la variable de tiempo debe ser tratada
como una variable entera. Por tanto, aunque en el pesudocódigo proporcionado se
generen las llegadas y el servicio utilizando valores reales, dichos valores obtenidos
deben ser transformados a enteros, redondeándolos al entero más próximo. Además, si
el valor que se obtiene al hacer las transformaciones correspondientes es 0, se
debe devolver 1, ya que si no, se generaría un suceso en el tiempo actual y, al
incrementar el tiempo en una unidad, ese suceso se quedaría en un tiempo anterior
al nuevo actual, y por tanto, nunca se podría llevar a cabo.

Una vez dicho esto, vamos a establecer

\end{document}

