\documentclass[11pt,a4paper]{article}
\usepackage[spanish,es-nodecimaldot]{babel}	% Utilizar español
\usepackage[utf8]{inputenc}					% Caracteres UTF-8
\usepackage{graphicx}						% Imagenes
\usepackage[hidelinks]{hyperref}			% Poner enlaces sin marcarlos en rojo
\usepackage{fancyhdr}						% Modificar encabezados y pies de pagina
\usepackage{float}							% Insertar figuras
\usepackage[textwidth=390pt]{geometry}		% Anchura de la pagina
\usepackage[nottoc]{tocbibind}				% Referencias (no incluir num pagina indice en Indice)
\usepackage{enumitem}						% Permitir enumerate con distintos simbolos
\usepackage[T1]{fontenc}					% Usar textsc en sections
\usepackage{amsmath}						% Símbolos matemáticos
\usepackage{tikz}
\usetikzlibrary{positioning, arrows, shapes}

\tikzset{
	->, % Hace que los arcos sean dirigidos
	>=stealth, % Hace que la punta de las flechas sean gruesas
	node distance=3cm % Distancia minima entre nodos
}

% Comando para poner el nombre de la asignatura
\newcommand{\asignatura}{Simulación de Sistemas}
\newcommand{\autor}{Vladislav Nikolov Vasilev}
\newcommand{\titulo}{Ejercicio 2}
\newcommand{\subtitulo}{Modelo Dinámico Discreto}

% Configuracion de encabezados y pies de pagina
\pagestyle{fancy}
\lhead{\autor{}}
\rhead{\asignatura{}}
\lfoot{Grado en Ingeniería Informática}
\cfoot{}
\rfoot{\thepage}
\renewcommand{\headrulewidth}{0.4pt}		% Linea cabeza de pagina
\renewcommand{\footrulewidth}{0.4pt}		% Linea pie de pagina

\begin{document}
\pagenumbering{gobble}

% Pagina de titulo
\begin{titlepage}

\begin{minipage}{\textwidth}

\centering

\includegraphics[scale=0.5]{img/ugr.png}\\

\textsc{\Large \asignatura{}\\[0.2cm]}
\textsc{GRADO EN INGENIERÍA INFORMÁTICA}\\[1cm]

\noindent\rule[-1ex]{\textwidth}{1pt}\\[1.5ex]
\textsc{{\Huge \titulo\\[0.5ex]}}
\textsc{{\Large \subtitulo\\}}
\noindent\rule[-1ex]{\textwidth}{2pt}\\[3.5ex]

\end{minipage}

\vspace{0.5cm}

\begin{minipage}{\textwidth}

\centering

\textbf{Autor}\\ {\autor{}}\\[2.5ex]
\textbf{Rama}\\ {Computación y Sistemas Inteligentes}\\[2.5ex]
\vspace{0.3cm}

\includegraphics[scale=0.3]{img/etsiit.jpeg}

\vspace{0.7cm}
\textsc{Escuela Técnica Superior de Ingenierías Informática y de Telecomunicación}\\
\vspace{1cm}
\textsc{Curso 2019-2020}
\end{minipage}
\end{titlepage}

\pagenumbering{arabic}
\tableofcontents
\thispagestyle{empty}				% No usar estilo en la pagina de indice

\newpage

\setlength{\parskip}{1em}

\section{\textsc{Descripción del problema}}

\section{\textsc{Grafo de sucesos}}

\begin{figure}[H]
\centering
\begin{tikzpicture}
\node [rectangle, draw] (inicio) {INICIO};
\node [circle, draw, text width=1.4cm, below=of inicio] (llegada) {Llegada Servidor};
\node [circle, draw, text width=1.8cm, right=of llegada] (iniA) {Comienzo servicio servidor A};
\node [circle, draw, text width=2cm, right=of iniA] (finA) {Fin servicio servidor A};
\node [circle, draw, text width=1.8cm, below=of finA] (iniB) {Comeinzo servicio servidor B};
\node [circle, draw, text width=2cm, left=of iniB] (finB) {Fin servicio servidor B};
\node [circle, draw, text width=1.3cm, left=of finB] (salida) {Salida sistema};
\draw	(inicio) edge[left] node{0} (llegada)
		(llegada) edge[in=60, out=30, loop, above] node{exp(1')}
		(llegada) edge[above] node{0} (iniA)
		(llegada) edge[below] node{$servA = libre$} (iniA)
		(iniA) edge[below] node{exp(0.8')} (finA)
		(finA) edge[bend right, above] node{$encolaA > 0$} (iniA)
		(finA) edge[bend right, below] node{0} (iniA)
		(finA) edge[left] node{0} (iniB)
		(finA) edge[right] node{$servB = libre$} (iniB)
		(iniB) edge[above] node{exp(1.2')} (finB)
		(finB) edge[bend right, below] node{$encolaB>0$} (iniB)
		(finB) edge[bend right, above] node{0} (iniB)
		(finB) edge[above] node{0} (salida)
;
\end{tikzpicture}
\caption{Versión inicial del grafo de sucesos.}
\end{figure}

\begin{figure}[H]
\centering
\begin{tikzpicture}
\node [rectangle, draw] (inicio) {INICIO};
\node [circle, draw, text width=1.4cm, below=of inicio] (llegada) {Llegada Servidor};
\node [circle, draw, text width=2cm, right=of llegada] (finA) {Fin servicio servidor A};
\node [circle, draw, text width=2cm, right=of finA] (finB) {Fin servicio servidor B};
\draw	(inicio) edge[left] node{0} (llegada)
		(llegada) edge[in=60, out=30, loop, above] node{exp(1')}
		(llegada) edge[above] node{exp(0.8')} (finA)
		(llegada) edge[below] node{$servA = libre$} (finA)
		(finA) edge[in=120, out=60, loop, above] node{$encolaA > 0$} (finA)
		(finA) edge[in=120, out=60, loop, below] node{exp(0.8')} (finA)
		(finA) edge[above] node{exp(1.2')} (finB)
		(finA) edge[below] node{$servB = libre$} (finB)
		(finB) edge[in=120, out=60, loop, above] node{$encolaB > 0$} (finB)
		(finB) edge[in=120, out=60, loop, below] node{exp(1.2')} (finB)
		
;
\end{tikzpicture}
\caption{Versión simplificada del grafo de sucesos.}
\end{figure}

\section{\textsc{Descripción del modelo}}

\section{\textsc{Experimentación y resultados}}

\end{document}

