\documentclass[11pt,a4paper]{article}
\usepackage[spanish,es-nodecimaldot]{babel}	% Utilizar español
\usepackage[utf8]{inputenc}					% Caracteres UTF-8
\usepackage{graphicx}						% Imagenes
\usepackage[hidelinks]{hyperref}			% Poner enlaces sin marcarlos en rojo
\usepackage{fancyhdr}						% Modificar encabezados y pies de pagina
\usepackage{float}							% Insertar figuras
\usepackage[textwidth=390pt]{geometry}		% Anchura de la pagina
\usepackage[nottoc]{tocbibind}				% Referencias (no incluir num pagina indice en Indice)
\usepackage{enumitem}						% Permitir enumerate con distintos simbolos
\usepackage[T1]{fontenc}					% Usar textsc en sections
\usepackage{amsmath}						% Símbolos matemáticos
\usepackage{tikz}
\usetikzlibrary{positioning, arrows, shapes}

\tikzset{
	->, % Hace que los arcos sean dirigidos
	>=stealth, % Hace que la punta de las flechas sean gruesas
	node distance=3cm % Distancia minima entre nodos
}

% Comando para poner el nombre de la asignatura
\newcommand{\asignatura}{Simulación de Sistemas}
\newcommand{\autor}{Vladislav Nikolov Vasilev}
\newcommand{\titulo}{Ejercicio 2}
\newcommand{\subtitulo}{Modelo Dinámico Discreto}

% Configuracion de encabezados y pies de pagina
\pagestyle{fancy}
\lhead{\autor{}}
\rhead{\asignatura{}}
\lfoot{Grado en Ingeniería Informática}
\cfoot{}
\rfoot{\thepage}
\renewcommand{\headrulewidth}{0.4pt}		% Linea cabeza de pagina
\renewcommand{\footrulewidth}{0.4pt}		% Linea pie de pagina

\begin{document}
\pagenumbering{gobble}

% Pagina de titulo
\begin{titlepage}

\begin{minipage}{\textwidth}

\centering

\includegraphics[scale=0.5]{img/ugr.png}\\

\textsc{\Large \asignatura{}\\[0.2cm]}
\textsc{GRADO EN INGENIERÍA INFORMÁTICA}\\[1cm]

\noindent\rule[-1ex]{\textwidth}{1pt}\\[1.5ex]
\textsc{{\Huge \titulo\\[0.5ex]}}
\textsc{{\Large \subtitulo\\}}
\noindent\rule[-1ex]{\textwidth}{2pt}\\[3.5ex]

\end{minipage}

\vspace{0.5cm}

\begin{minipage}{\textwidth}

\centering

\textbf{Autor}\\ {\autor{}}\\[2.5ex]
\textbf{Rama}\\ {Computación y Sistemas Inteligentes}\\[2.5ex]
\vspace{0.3cm}

\includegraphics[scale=0.3]{img/etsiit.jpeg}

\vspace{0.7cm}
\textsc{Escuela Técnica Superior de Ingenierías Informática y de Telecomunicación}\\
\vspace{1cm}
\textsc{Curso 2019-2020}
\end{minipage}
\end{titlepage}

\pagenumbering{arabic}
\tableofcontents
\thispagestyle{empty}				% No usar estilo en la pagina de indice

\newpage

\setlength{\parskip}{1em}

\section{\textsc{Descripción del problema}}

Un sistema de colas consta de 2 servidores (A y B) dispuestos en serie. Los clientes
que acceden al sistema son primero atendidos por el servidor A. Una vez que acaba
la atención en el servidor A, pasan a ser atendidos por el servidor B, y cuando terminan
de ser atendidos, salen del sistema. Cada servidor tiene una cola de espera FIFO. Todos
los tiempos siguen una distribución exponencial. Para las llegadas, dicha distribución tiene
una media de 1 minuto. Para el tiempo de servicio en el servidor A se tiene una media de $0.8$
minutos, mientras que para el servidor B se tiene una media de $1.2$ minutos.

El objetivo es construir un modelo de simulación (en este caso, un modelo dinámico discreto)
el cuál permita simular el sistema anteriormente descrito y que además, permita calcular el tiempo de estancia
medio de los clientes en el sistema. Adicionalmente, se quiere determinar hasta cuánto tiempo
habría que reducir el tiempo de servicio del servidor B, dejando el mismo tiempo de servicio para el servidor
A, para conseguir un tiempo medio de estancia inferior a 10 minutos.

\section{\textsc{Grafo de sucesos}}

Una vez que hemos hecho la descripción del problema, lo primero que nos interesa hacer es determinar
los sucesos relevantes que tenemos que modelizar. Si nos paramos a analizar la secuencia de sucesos
que se tienen que seguir para atender a un cliente de principio a fin, nos encontramos con lo siguiente:

\begin{enumerate}
	\item Lo primero que se produce es la llegada del cliente al sistema. Por tanto, debe existir
	un suceso que represente dicha llegada al sistema.
	\item Una vez que el cliente ha llegado, si el servidor A está libre, se inicia la atención
	del cliente. En caso contrario, el cliente debe esperar su turno en la cola. Por tanto, necesitamos
	un suceso que represente el inicio de atención en el servidor A.
	\item Ya que hemos decidido modelizar el inicio de la atención en el servidor A, también necesitamos
	modelizar el final de atención. Puede suceder que una vez que se termine de atender el cliente actual
	queden otros en la cola, con lo cuál se podría pasar a atender el primero. En caso de no haber ninguno,
	el servidor quedaría libre, de forma que el siguiente cliente no necesitaría esperar en la cola para poder
	pasar a ser atendido.
	\item Lo siguiente que pasaría sería que el cliente pasa a ser atendido por el servidor B, siempre
	y cuando este esté libre. En caso de que no lo esté, deberá quedarse esperando en la cola hasta
	que le toque. Por tanto, necesitamos un suceso que represente el inicio de atención en el servidor B.
	\item Tal y como hicimos antes, necesitaremos un suceso que represente el fin de atención en el
	servidor B. De aquí, tal y como pasaba antes, puede suceder que se pase a atender el siguiente cliente
	si hay alguien esperando en la cola o que el servidor se quede libre.
	\item Finalmente, una vez que ha finalizado la atención en el servidor B, el cliente sale del sistema.
	Por tanto, necesitamos un suceso que represente la salida del sistema.
\end{enumerate}

Con los seis sucesos anteriormente descritos, podemos construir el grafo de sucesos. Adicionalmente,
necesitaríamos un suceso extra que genere la primera llegada. El grafo resultante se puede ver a continuación:

\begin{figure}[H]
\centering
\begin{tikzpicture}
\node [rectangle, draw] (inicio) {INICIO};
\node [circle, draw, text width=1.4cm, below=of inicio] (llegada) {Llegada Servidor};
\node [circle, draw, text width=1.8cm, right=of llegada] (iniA) {Comienzo servicio servidor A};
\node [circle, draw, text width=2cm, right=of iniA] (finA) {Fin servicio servidor A};
\node [circle, draw, text width=1.8cm, below=of finA] (iniB) {Comeinzo servicio servidor B};
\node [circle, draw, text width=2cm, left=of iniB] (finB) {Fin servicio servidor B};
\node [circle, draw, text width=1.3cm, left=of finB] (salida) {Salida sistema};
\draw	(inicio) edge[left] node{exp(1')} (llegada)
		(llegada) edge[in=60, out=30, loop, above] node{exp(1')}
		(llegada) edge[above] node{0} (iniA)
		(llegada) edge[below] node{$servA = libre$} (iniA)
		(iniA) edge[below] node{exp(0.8')} (finA)
		(finA) edge[bend right, above] node{$encolaA > 0$} (iniA)
		(finA) edge[bend right, below] node{0} (iniA)
		(finA) edge[left] node{0} (iniB)
		(finA) edge[right] node{$servB = libre$} (iniB)
		(iniB) edge[above] node{exp(1.2')} (finB)
		(finB) edge[bend right, below] node{$encolaB>0$} (iniB)
		(finB) edge[bend right, above] node{0} (iniB)
		(finB) edge[above] node{0} (salida)
;
\end{tikzpicture}
\caption{Versión inicial del grafo de sucesos.}
\end{figure}

El grafo anterior puede ser simplificado ya que hay un conjunto de sucesos a los que solo
entran arcos con duración 0. Estos nodos son \textbf{Comienzo servicio servidor A},
\textbf{Comienzo servicio servidor B} y \textbf{Salida sistema}. Si lo simplificamos,
eliminando dichos nodos y ajustando los enlaces que salen de los nodos eliminados,
obtenemos el siguiente grafo:

\begin{figure}[H]
\centering
\begin{tikzpicture}
\node [rectangle, draw] (inicio) {INICIO};
\node [circle, draw, text width=1.4cm, below=of inicio] (llegada) {Llegada Servidor};
\node [circle, draw, text width=2cm, right=of llegada] (finA) {Fin servicio servidor A};
\node [circle, draw, text width=2cm, right=of finA] (finB) {Fin servicio servidor B};
\draw	(inicio) edge[left] node{exp(1')} (llegada)
		(llegada) edge[in=60, out=30, loop, above] node{exp(1')}
		(llegada) edge[above] node{exp(0.8')} (finA)
		(llegada) edge[below] node{$servA = libre$} (finA)
		(finA) edge[in=120, out=60, loop, above] node{$encolaA > 0$} (finA)
		(finA) edge[in=120, out=60, loop, below] node{exp(0.8')} (finA)
		(finA) edge[above] node{exp(1.2')} (finB)
		(finA) edge[below] node{$servB = libre$} (finB)
		(finB) edge[in=120, out=60, loop, above] node{$encolaB > 0$} (finB)
		(finB) edge[in=120, out=60, loop, below] node{exp(1.2')} (finB)
		
;
\end{tikzpicture}
\caption{Versión simplificada del grafo de sucesos.}
\end{figure}

Vemos que este grafo está compuesto por tres sucesos y el suceso de inicio. Las conexiones
que salían anteriormente de los nodos eliminados se han conservado, de forma que no se
ha perdido ninguna transición.

\section{\textsc{Descripción del modelo}}

Una vez que hemos visto el grafo de sucesos, vamos a describir algunos aspectos
fundamentales del modelo de cara a la implementación de este.

\subsection{Variables de interés}

\subsection{Rutinas de interés}

\subsection{Estructura y composición de la lista de sucesos}

\section{\textsc{Experimentación y resultados}}

\end{document}

