\documentclass[11pt,a4paper]{report}
\usepackage[spanish,es-nodecimaldot]{babel}	% Utilizar español
\usepackage[utf8]{inputenc}					% Caracteres UTF-8
\usepackage{graphicx}						% Imagenes
\usepackage[hidelinks]{hyperref}			% Poner enlaces sin marcarlos en rojo
\usepackage{fancyhdr}						% Modificar encabezados y pies de pagina
\usepackage{float}							% Insertar figuras
\usepackage[textwidth=390pt]{geometry}		% Anchura de la pagina
\usepackage[nottoc]{tocbibind}				% Referencias (no incluir num pagina indice en Indice)
\usepackage{enumitem}						% Permitir enumerate con distintos simbolos
\usepackage[T1]{fontenc}					% Usar textsc en sections
\usepackage{amsmath}						% Símbolos matemáticos
\usepackage{listings}
\usepackage{longtable}
\usepackage{subcaption}

% Comando para poner el nombre de la asignatura
\newcommand{\asignatura}{Simulación de Sistemas}
\newcommand{\autor}{Vladislav Nikolov Vasilev}
\newcommand{\titulo}{PRÁCTICA 1}
\newcommand{\subtitulo}{Diferentes Modelos de Simulación}

% Configuracion de encabezados y pies de pagina
\pagestyle{fancy}
\lhead{\autor{}}
\rhead{\asignatura{}}
\lfoot{Grado en Ingeniería Informática}
\cfoot{}
\rfoot{\thepage}
\renewcommand{\headrulewidth}{0.4pt}		% Linea cabeza de pagina
\renewcommand{\footrulewidth}{0.4pt}		% Linea pie de pagina

\begin{document}
\pagenumbering{gobble}

% Pagina de titulo
\begin{titlepage}

\begin{minipage}{\textwidth}

\centering

\includegraphics[scale=0.5]{img/ugr.png}\\

\textsc{\Large \asignatura{}\\[0.2cm]}
\textsc{GRADO EN INGENIERÍA INFORMÁTICA}\\[1cm]

\noindent\rule[-1ex]{\textwidth}{1pt}\\[1.5ex]
\textsc{{\Huge \titulo\\[0.5ex]}}
\textsc{{\Large \subtitulo\\}}
\noindent\rule[-1ex]{\textwidth}{2pt}\\[3.5ex]

\end{minipage}

\vspace{0.5cm}

\begin{minipage}{\textwidth}

\centering

\textbf{Autor}\\ {\autor{}}\\[2.5ex]
\textbf{Rama}\\ {Computación y Sistemas Inteligentes}\\[2.5ex]
\vspace{0.3cm}

\includegraphics[scale=0.3]{img/etsiit.jpeg}

\vspace{0.7cm}
\textsc{Escuela Técnica Superior de Ingenierías Informática y de Telecomunicación}\\
\vspace{1cm}
\textsc{Curso 2018-2019}
\end{minipage}
\end{titlepage}

\pagenumbering{arabic}
\tableofcontents
\thispagestyle{empty}				% No usar estilo en la pagina de indice

\newpage

\setlength{\parskip}{1em}

\chapter{Mi primer modelo de Montecarlo}

\section{Pruebas iniciales con el modelo}

Inicialmente, se ha probado el modelo para ver cómo funcionaba. Se ha ejecutado el modelo una serie de
veces (10 para ser más precisos), y se han almacenado los resultados. Estos resultados pueden verse a
continuación:

\begin{table}[H]
\centering
\begin{tabular}{c|c}
\textbf{Mejor posición inicial ($c$)} & \textbf{Mejor distancia} \\ \hline
95                              & 6.525780                 \\
94                              & 6.472190                 \\
94                              & 6.513340                 \\
94                              & 6.531020                 \\
94                              & 6.498630                 \\
93                              & 6.512340                 \\
94                              & 6.469870                 \\
94                              & 6.487920                 \\
94                              & 6.486590                 \\
94                              & 6.478660                
\end{tabular}
\caption{Resultados de mejor posición inicial y mejor distancia para 10 ejecuciones.}
\label{aparc-tabla}
\end{table}

Como se puede apreciar, parece que hay cierta semejanza entre los resultados obtenidos en cada
ejecución. Se puede ver que el valor de $c$ obtenido es muy similar en casi todos los casos, siendo
la moda $c = 94$, y la media un valor muy próximo a éste. Los valores de la mejor distancia también
son muy próximos entre sí, ya que todos ellos rondan, aproximadamente, las 6.5 plazas. Por tanto, el
modelo es capaz de producir unos resultados muy similares a pesar de que utiliza cierta ``aleatoriedad''.

Si estudiamos cómo evoluciona el valor de la distancia al destino a medida que cambia el valor de $c$, nos
encontramos con la siguiente gráfica:

\begin{figure}[H]
\centering
\includegraphics[scale=0.6]{img/c-dist.png}
\caption{Evolución del valor de la distancia al destino en función del valor de la posición inicial.}
\label{aparc-grafica}
\end{figure}

Como se puede ver, existe una tendencia a que, cuanto más cerca de la posición a la que se quiera llegar
se comienza a buscar sitio (es decir, cuanto más alto sea el valor de $c$), menor será la distancia hasta
el destino. Esto es completamente lógico, ya que al comenzar a buscar sitio a partir de una posición
muy lejana al destino, mayor será la distancia hasta éste en caso de que se encuentre una plaza libre. Teniendo
en cuenta que se escoge la primera plaza libre, ésta puede quedar muy lejos del destino, que es lo que se puede
apreciar en la gráfica.

El valor ideal de $c$, con las condiciones en las que estamos, parece ser $c = 94$. A partir de ahí, se puede ver
que existe un ligero incremento en la distancia al destino, posiblemente porque se aparque más lejos debido a que
no se encuentre una plaza libre en posiciones más cercanas al destino.

Por tanto, a vista de los resultados que hemos obtenido, podemos afirmar con bastante certeza que la plaza ideal
a partir de la que empezar sitio para aparcar es aproximadamente la 94.

\section{Experimentación con los parámetros}

Se han realizado una serie de experimentaciones con los parámetros con los que se puede llamar al programa.
A continuación, se muestran algunos de los resultados obtenidos.

\subsection{Modificación de la posición destino}

Se ha probado a modificar la posición destino, estableciendo que $x = 150$, y se han realizado 10 ejecuciones.
A continuación, se pueden ver los resultados obtenidos:

\begin{table}[H]
\centering
\begin{tabular}{c|c}
\textbf{Mejor posición inicial ($c$)} & \textbf{Mejor distancia} \\ \hline
144                              & 6.438900                 \\
144                              & 6.438900                 \\
144                              & 6.438900                 \\
143                              & 6.472200                 \\
143                              & 6.472200                 \\
143                              & 6.472200                 \\
143                              & 6.472200                 \\
143                              & 6.472200                 \\
143                              & 6.472200                 \\
143                              & 6.358800                
\end{tabular}
\caption{Resultados de mejor posición inicial y mejor distancia para 10 ejecuciones con
con posición destino $x = 150$.}
\label{aparc-150-tabla}
\end{table}

\begin{figure}[H]
\centering
\includegraphics[scale=0.47]{img/c-dist.png}
\caption{Evolución del valor de la distancia al destino en función del valor de la posición inicial para $x = 150$.}
\label{aparc-150-grafica}
\end{figure}

Los resultados obtenidos en la tabla \ref{aparc-150-tabla} son muy parecidos a los que obtuvimos en \ref{aparc-tabla},
solo que los valores de $c$ han cambiado, aunque siguen unos patrones parecisos. En este caso, la moda parece ser 143,
y la media tiende a ese valor. Las mejores distancias están también muy próximas, y no se ve mucha disparidad.

Además, tal y como se hizo en el caso anterior, se ha obtenido una gráfica que muestra la evolución de la distancia media
en función del valor de $c$, la cuál se puede ver en la figura \ref{aparc-150-grafica}. Como se puede observar, sigue
un patrón muy parecido al que se puede ver en la figura \ref{aparc-grafica}, así que parece que no es un parámetro
que influya demasiado por sí solo.

\subsection{Modificación de la probabilidad de ocupación}

Se ha probado a variar la probabilidad de ocupación para ver cómo es afectada la salida. A continuación, se pueden ver
los resultados que se han obtenido tras realizar 11 ejecuciones:

\begin{table}[H]
\begin{tabular}{c|c|c}
\textbf{Probabilidad de ocupación} & \textbf{Mejor posición inicial ($c$)} & \textbf{Mejor distancia} \\ \hline
0.01                               & 99                              & 0.008400                 \\
0.1                                & 99                              & 0.100000                 \\
0.2                                & 99                              & 0.213100                 \\
0.3                                & 99                              & 0.344800                 \\
0.4                                & 99                              & 0.495100                 \\
0.5                                & 98                              & 0.744800                 \\
0.6                                & 98                              & 1.074900                 \\
0.7                                & 98                              & 1.659600                 \\
0.8                                & 97                              & 2.901000                 \\
0.9                                & 94                              & 6.477300                 \\
0.99                               & 36                              & 67.957802               
\end{tabular}
\caption{Valores de la mejor posición inicial y distancia en función de la probabilidad de ocupación.}
\label{aparc-tabla-prob}
\end{table}

Es interesante ver cómo, a medida que va incrementando la probabilidad de ocupación, los valores de $c$ y de distancia
van cambiando, y se van haciendo cada vez peores. Se puede ver como el valor de $c$ va disminuyendo a medida que va aumentando
la probabilidad, lo cuál afecta directamente a la distancia, aumentándola proporcionalmente cada vez que disminuye el valor de
$c$. Todo esto se debe a que, cuanto menor sea la probabilidad de ocupación, más cerca del destino se podrá encontrar un sitio.
Todo esto se puede ver gráficamente en la siguiente figura:

\begin{figure}[H]
\centering
\includegraphics[scale=0.6]{img/aparc-prob-3d.png}
\caption{Representación 3D de la distancia media al destino en función de la probabilidad de ocupación y
la posición inicial.}
\label{aparc-3d-prob}
\end{figure}

Si se prueba con valores muy extremos, como por ejemplo con probabilidad de ocupación de 0.999, se obtiene el
siguiente error:

\begin{figure}[H]
\centering
\includegraphics[scale=0.3]{img/aparc-prob-error.png}
\caption{Error al ejecutar el programa \textit{aparcamiento} con 0.999 de probabilidad de ocupación.}
\end{figure}

Esto se debe a que no se consigue encontrar sitio libre y se supera el tamaño del vector que representa las posiciones,
lo cuál genera un fallo de segmentación al intentar acceder a posiciones no válidas de memoria.

\subsection{Modificación del nivel de visión}

Vamos a intentar variar ahora el nivel de visión para ver cómo son afectados los resultados. Estos son los resultados
obtenidos tras haber realizado 11 ejecuciones:

\begin{table}[H]
\begin{tabular}{c|c|c}
\textbf{Alcance de visión} & \textbf{Mejor posición inicial ($c$)} & \textbf{Mejor distancia} \\ \hline
2                          & 94                              & 6.480800                 \\
4                          & 94                              & 6.339700                 \\
6                          & 94                              & 6.069100                 \\
8                          & 93                              & 5.706100                 \\
10                         & 92                              & 5.505300                 \\
15                         & 90                              & 5.244000                 \\
20                         & 86                              & 4.954400                 \\
25                         & 83                              & 4.890900                 \\
30                         & 82                              & 4.778700                 \\
50                         & 72                              & 4.687100                 \\
100                        & 80                              & 4.624000                
\end{tabular}
\caption{Valores de la mejor posición inicial y distancia en función del alcance de visión.}
\label{aparc-tabla-vis}
\end{table}

Como se puede ver, aquí también al aumentar el valor del alcance de visión, el valor de $c$ disminuye. Sin embargo,
a diferencia del caso anterior, la mejor distancia media va disminuyendo, posiblemente debido a que se tenga más
conocimiento de las posiciones venideras. En el caso extremo en el que se ven todas las plazas de aparcamiento,
se ve que la distancia media es la mínima. Todo esto se puede ver en la siguiente figura:

\begin{figure}[H]
\centering
\includegraphics[scale=0.49]{img/aparc-vis-3d.png}
\caption{Representación 3D de la distancia media al destino en función del alcance de visión y
la posición inicial.}
\label{aparc-3d-vis}
\end{figure}

\subsection{Modificación de la visión y la probabilidad de ocupación}

En este último apartado, se ha querido ver qué pasa si se modifican dos variables a la vez. Para ello, se van a
modificar tanto el alcance de visión como la probabilidad de ocupación.

Los resultados de haber probado 9 combinaciones de parámetros se pueden ver en la siguiente tabla:

\begin{table}[H]
\resizebox{\columnwidth}{!}{%
\begin{tabular}{c|c|c|c}
\textbf{\begin{tabular}[c]{@{}c@{}}Probabilidad\\ de ocupación\end{tabular}} & \textbf{Alcance de visión} & \textbf{Mejor posición inicial  ($c$)} & \textbf{Mejor distancia} \\ \hline
0.4                                                                          & 5                          & 98                              & 0.480700                 \\
0.4                                                                          & 10                         & 94                              & 0.469800                 \\
0.4                                                                          & 15                         & 90                              & 0.459100                 \\
0.5                                                                          & 5                          & 97                              & 0.660100                 \\
0.5                                                                          & 10                         & 96                              & 0.661500                 \\
0.5                                                                          & 15                         & 90                              & 0.655300                 \\
0.6                                                                          & 5                          & 96                              & 1.472600                 \\
0.6                                                                          & 10                         & 95                              & 1.362600                 \\
0.6                                                                          & 15                         & 91                              & 1.364300                
\end{tabular}
}%
\caption{Valores de la mejor posición inicial y distancia en función de la probabilidad de ocupación y el alcance
de visión.}
\label{aparc-tabla-prob-vis}
\end{table}

Como se puede ver, aún con la misma probabilidad de ocupación, el hecho de tener un mayor alcance de visión permite
disminuir la mejor distancia media al destino, lo cuál sugiere que las suposiciones anteriores sobre que el alcance de
visión permite tener más conocimiento sobre el problema son correctas. También se puede ver un claro patrón en el que,
si se incrementa solo la probabilidad de ocupación sin cambiar el alcance de visión, la mejor distancia media empeora,
pero al incrementar el alcance, ésta disminuye, tal y como se puede observar en las tablas \ref{aparc-tabla-prob} y
\ref{aparc-tabla-vis}.

\newpage

\chapter{Mi primer modelo de simulación discreto}

\section{Estudio experimental del número de repuestos}

Se va a realizar un estudio experimental del número de repuestos mínimo que se necesita. Para ello, se va a ejecutar
el programa una serie de veces, con un número de piezas de repuesto y de repeticiones diferente. En concreto, se van
a probar 5, 7, 9, 11 y 12 piezas, y 1, 10, 100, 500 y 1000 repeticiones. A continuación se pueden ver los resultados:

\begin{longtable}{c|c|c}
\textbf{\begin{tabular}[c]{@{}c@{}}Nº piezas\\ de repuesto\end{tabular}} & \textbf{Nº de repeticiones} & \textbf{\begin{tabular}[c]{@{}c@{}}Media del \% de\\ tiempo de desprotección\end{tabular}} \\ \hline
5                                                                        & 1                           & 37.1069                                                                                    \\
7                                                                        & 1                           & 11.7863                                                                                    \\
9                                                                        & 1                           & 3.26835                                                                                    \\
11                                                                       & 1                           & 0.554182                                                                                   \\
12                                                                       & 1                           & 0                                                                                          \\ \hline
5                                                                        & 10                          & 31.9152                                                                                    \\
7                                                                        & 10                          & 14.5627                                                                                    \\
9                                                                        & 10                          & 2.8476                                                                                     \\
11                                                                       & 10                          & 0.944855                                                                                   \\
12                                                                       & 10                          & 0.410312                                                                                   \\ \hline
5                                                                        & 100                         & 38.0701                                                                                    \\
7                                                                        & 100                         & 16.1558                                                                                    \\
9                                                                        & 100                         & 4.23885                                                                                    \\
11                                                                       & 100                         & 1.10964                                                                                    \\
12                                                                       & 100                         & 0.593476                                                                                   \\ \hline
5                                                                        & 500                         & 36.92                                                                                      \\
7                                                                        & 500                         & 15.3941                                                                                    \\
9                                                                        & 500                         & 4.94555                                                                                    \\
11                                                                       & 500                         & 1.05448                                                                                    \\
12                                                                       & 500                         & 0.394317                                                                                   \\ \hline
5                                                                        & 1000                        & 37.2712                                                                                    \\
7                                                                        & 1000                        & 15.5255                                                                                    \\
9                                                                        & 1000                        & 4.83779                                                                                    \\
11                                                                       & 1000                        & 0.943597                                                                                   \\
12                                                                       & 1000                        & 0.435409       
\end{longtable}

Si analizamos los resultados detenidamente, podemos ver que, al aumentar el número de repeticiones, los resultados
toman unos valores que se acercan más al comportamiento promedio. Esto se debe a que en el modelo existen ciertas
variables aleatorias. Por tanto, sacar conclusiones a partir de una única ejecución no sería algo válido. Habría
que hacer un gran número de repeticiones (como por ejemplo 500, 1000 o más) para sacar conclusiones verdaderamente válidas.

Teniendo en cuenta lo dicho anteriormente y a la vista de todos los resultados que se han obtenido,
se puede concluir que el número mínimo de piezas de repuesto que se necesitaría para que el
tiempo de desprotección total sea inferior al 1\% son \textbf{12 piezas}. Podemos concluir esto porque, viendo
los resultados que se han obtenido al variar el número de repeticiones, en el caso de 12 piezas nunca se pasa de ese
umbral. El hecho de tener 11 piezas es suficiente en algunos casos, pero en otros, se supera ese porcentaje de tiempo.
Por tanto, aunque podríamos decir que el valor óptimo de piezas de repuesto está entre 11 y 12, vamos a quedarnos con
las 12, tal y cómo hemos dicho anteriormente, debido a que ofrece más ``seguridad'' tener una pieza más.

\section{Experimentación con los parámetros}

Habiendo determinado el número mínimo de componentes de repuesto necesarios, vamos a ver cómo modificando los valores
de algunos de los parámetros cambian los resultados obtenidos.

\subsection{Modificación del tiempo de reparación}

Vamos a disminuir el tiempo de reparación para ver cuál sería en este caso el mínimo número de componentes que
necesitaríamos. Para ello, vamos a cambiar los tiempos para que se tarde entre 5 y 15 días, siguiendo una distribución
uniforme. Para evitar tener tantos datos como antes, vamos a considerar solo 100, 500 y 1000 repeticiones, ya que
hemos visto que son una buena aproximación porque se trata de un resultado medio.

A continuación se pueden ver los resultados:

\begin{table}[H]
\begin{tabular}{c|c|c}
\textbf{\begin{tabular}[c]{@{}c@{}}Nº piezas\\ de repuesto\end{tabular}} & \textbf{Nº de repeticiones} & \textbf{\begin{tabular}[c]{@{}c@{}}Media del \% de\\ tiempo de desprotección\end{tabular}} \\ \hline
5                                                                        & 100                         & 3.74915                                                                                    \\
7                                                                        & 100                         & 0.412021                                                                                   \\
9                                                                        & 100                         & 0.0101209                                                                                  \\
11                                                                       & 100                         & 0                                                                                          \\
12                                                                       & 100                         & 0                                                                                          \\ \hline
5                                                                        & 500                         & 3.7748                                                                                     \\
7                                                                        & 500                         & 0.354648                                                                                   \\
9                                                                        & 500                         & 0.026469                                                                                   \\
11                                                                       & 500                         & 0.000546791                                                                                \\
12                                                                       & 500                         & 0                                                                                          \\ \hline
5                                                                        & 1000                        & 3.56635                                                                                    \\
7                                                                        & 1000                        & 0.393049                                                                                   \\
9                                                                        & 1000                        & 0.02624                                                                                    \\
11                                                                       & 1000                        & 0.000315176                                                                                \\
12                                                                       & 1000                        & 0.000970965                                                                               
\end{tabular}
\end{table}

Como se puede ver claramente, parece ser que con un número de 6-7 piezas de repuesto se puede conseguir que el porcentaje
del tiempo de desprotección sea inferior al 1\%. Este número es inferior al que teníamos antes, el cuál rondaba las 12 piezas,
lo cuál es bastante bueno, ya que implica invertir menos en piezas, pero por otra parte, implica invertir más en las reparaciones.
Si estas no son muy caras (y teniendo en cuenta que las piezas de por sí son caras), sería más recomendable buscar
una reparación más rápida que disponer de un número mayor de piezas.

\subsection{Modificación del tiempo de vida}

Vamos a aumentar el tiempo de vida de los componentes hasta 50 días de media, siguiendo una distribución exponencial, para
ver como esta modificación afecta a la cantidad mínima de componentes de repuesto que se necesitarían. Para ello, se ha
seguido un proceso similar al que se ha llevado a cabo en la sección anterior.

A continuación se pueden ver los resultados que se han obtenido:

\begin{table}[H]
\begin{tabular}{c|c|c}
\textbf{\begin{tabular}[c]{@{}c@{}}Nº piezas\\ de repuesto\end{tabular}} & \textbf{Nº de repeticiones} & \textbf{\begin{tabular}[c]{@{}c@{}}Media del \% de\\ tiempo de desprotección\end{tabular}} \\ \hline
5                                                                        & 100                         & 2.6056                                                                                     \\
7                                                                        & 100                         & 0.163214                                                                                   \\
9                                                                        & 100                         & 0.0138627                                                                                  \\
11                                                                       & 100                         & 0                                                                                          \\
12                                                                       & 100                         & 0                                                                                          \\ \hline
5                                                                        & 500                         & 2.34215                                                                                    \\
7                                                                        & 500                         & 0.163914                                                                                   \\
9                                                                        & 500                         & 0.0128837                                                                                  \\
11                                                                       & 500                         & 0                                                                                          \\
12                                                                       & 500                         & 0                                                                                          \\ \hline
5                                                                        & 1000                        & 2.4367                                                                                     \\
7                                                                        & 1000                        & 0.207335                                                                                   \\
9                                                                        & 1000                        & 0.0154164                                                                                  \\
11                                                                       & 1000                        & 0                                                                                          \\
12                                                                       & 1000                        & 0                                                                                         
\end{tabular}
\end{table}

En este caso, con esta modificación, se obtienen unos resultados muy parecidos al caso anterior, ya que parece que el
número mínimo de componentes de repuesto ronda las 6-7 piezas. No obstante, aunque los resultados sean similares, parece
que el porcentaje del tiempo de desprotección es menor en este caso, lo cuál parece indicar que tener componentes más
duraderos es mejor que tener unos tiempos de reparación más pequeños. En la siguiente gráfica, se pueden ver
las diferencias:

\begin{figure}[H]
\centering
\includegraphics[scale=0.55]{img/radar.png}
\caption{Gráfica comparativa para 1000 repeticiones modificando el tiempo de reparación y el tiempo medio de vida.}
\end{figure}

Como se puede ver, existe cierta diferencia, aunque a medida que se van aumentando el número de piezas, ésta se hace
menos notable. Esta diferencia puede deberse a que el valor escogido como tiempo medio de vida sea bastante más alto que
los valores que se habían escogido como extremos del intervalo del tiempo de reparación; es decir, que se ha variado más
el tiempo medio de vida que el de reparación. Pero, parece que en este caso, si tuvieramos que escoger alguna de las
dos opciones sin considerar nada más, parece que sería más recomendable escoger piezas con una mayor duración, ya que
ofrecen una mayor protección.

\subsection{Conclusión}

Parece que tener unos tiempos de reparación más bajos y disponer de componentes más duraderos permite reducir el número
de componentes de repuesto necesarios. Parece que estos dos parámetros tienen bastante influencia en este número, así que,
en caso de querer determinar cuántos componentes necesitaríamos como mínimo, tenemos que considerar, además del precio de
éstos, cuánto tiempo se tardaría en reparar y el tiempo de vida medio de éstos. Dependiendo de cuáles sean los valores
de estos tiempos, nos interesará tener un mayor o un menor número de componentes.

También hay que considerear que, al ser los procesos de reparación más rápidos, éstos pueden ser más caros. Lo mismo pasa
con los componentes de mayor durabilidad: sus precios pueden ser mayores que unos que duran bastante menos tiempo. Por tanto,
en una situación real, también tenemos que tener en cuenta el factor económico para tomar una decisión con los datos que
nos ofrece la simulación, ya que estos factores no se han tenido en cuenta a la hora de desarrollar el modelo.

\newpage

\chapter{Mi primer modelo de simulación continuo}

\section{Búsqueda del punto de equilibrio}

Vamos a comenzar buscando un punto de equilibrio, en el que ninguno de los tipos de peces se extinga. Para ello, vamos a
simular la vida en el lago durante 20000 días, y vamos a ir probando distintas cantidades iniciales de peces de cada tipo
hasta que demos con un número en el que haya equilibrio.

Al probar con 1000 peces pequeños y 10 grandes, se han obtenido los siguientes resultados:

\begin{figure}[H]
\centering
\begin{minipage}{.5\textwidth}
  \centering
  \includegraphics[scale=0.4]{img/peces-1000-1.png}
  \caption{Evolución del número de peces pequeños en el tiempo.}
  \label{fig:peces-1000-p}
\end{minipage}%
\begin{minipage}{.5\textwidth}
  \centering
  \includegraphics[scale=0.4]{img/peces-1000-2.png}
  \caption{Evolución del número de peces grandes en el tiempo.}
  \label{fig:peces-1000-g}
\end{minipage}
\end{figure}

Como se puede ver, no se alcanza una situación de equilibrio. En la figura \ref{fig:peces-1000-p} se puede ver como, al pasar
solo unos pocos días, éstos se extinguen. Además, se puede ver en la figura \ref{fig:peces-1000-g} como el número de peces
grandes comienza a disminuir con el tiempo. Por tanto, de aquí deducimos que necesitamos un mayor número de peces pequeños
inicialmente, ya que se extinguen demasiado rápido.

Al probar ahora con 1500 peces pequeños y 10 grandes, se han obtenido los siguientes resultados:

\begin{figure}[H]
\centering
\begin{minipage}{.5\textwidth}
  \centering
  \includegraphics[scale=0.4]{img/peces-1500-1.png}
  \caption{Evolución del número de peces pequeños en el tiempo.}
  \label{fig:peces-1500-p}
\end{minipage}%
\begin{minipage}{.5\textwidth}
  \centering
  \includegraphics[scale=0.4]{img/peces-1500-2.png}
  \caption{Evolución del número de peces grandes en el tiempo.}
  \label{fig:peces-1500-g}
\end{minipage}
\end{figure}

Se puede ver que en este caso las dos poblaciones de peces se han estabilizado. Se puede ver claramente como se llega
a un punto de equilibrio en el que los números de peces no varían. Este tiempo es poco más de 2500 días. Se puede ver,
además, que la población de peces pequeños crece mucho más que la de peces grandes.

\begin{figure}[H]
\centering
\includegraphics[scale=0.52]{img/peces-1500-3.png}
\caption{Evolución del número de peces pequeños y grandes.}
\label{fig:peces-peq-grand}
\end{figure}

Si estudiamos como evolucionan las dos poblaciones, tal y como se puede ver en la figura \ref{fig:peces-peq-grand}, podemos
ver que tiene una forma que recuerda a una espiral. Esto se debe a que, al principio, la población de peces pequeños crece
desmesuradamente (como se puede ver en el pico inicial). Sin embargo, después de un tiempo, el número de peces pequeños
comienza a disminuir, debido a que el número de peces grandes va aumentando. Posteriormente, el número de peces grandes
comienza a disminuir, debido a que hay pocos peces pequeños, y así sucesivamente. De esta forma se crea la espiral que
se ve en la figura, hasta que se llega al punto de equilibrio, en el que ya no se producen cambios.

\section{Estudio de una campaña de pesca}

Vamos a probar qué pasa si capturamos un determinado número de peces grandes cada cierto tiempo. Para ello, vamos a suponer
que, partiendo de una situación en la que sabemos que hay equilibrio (gracias a los resultados del apartado anterior), se
captura un número determinado de peces grandes. Esta captura se llevará a mitad de simulación (es decir, al cabo de uno 10000
días). Se harán capturas tales que queden la mitad de los peces, un tercio y un cuarto.

A continuación, se pueden ver los resultados obtenidos:

\begin{figure}[H]
\centering
\begin{minipage}{.5\textwidth}
  \centering
  \includegraphics[scale=0.4]{img/peces-p-mitad.png}
  \subcaption{Evolución del número de peces pequeños.}
  \label{fig:peces-p-mitad}
\end{minipage}%
\begin{minipage}{.5\textwidth}
  \centering
  \includegraphics[scale=0.4]{img/peces-g-mitad.png}
  \subcaption{Evolución del número de peces grandes.}
  \label{fig:peces-g-mitad}
\end{minipage}
\caption{Evolución del número de peces de cada población en el tiempo dejando un cuarto de los grandes.}
\end{figure}

\begin{figure}[H]
\centering
\begin{minipage}{.5\textwidth}
  \centering
  \includegraphics[scale=0.4]{img/peces-p-tercio.png}
  \subcaption{Evolución del número de peces pequeños.}
  \label{fig:peces-p-tercio}
\end{minipage}%
\begin{minipage}{.5\textwidth}
  \centering
  \includegraphics[scale=0.4]{img/peces-g-tercio.png}
  \subcaption{Evolución del número de peces grandes.}
  \label{fig:peces-g-tercio}
\end{minipage}
\caption{Evolución del número de peces de cada población en el tiempo dejando un tercio de los grandes.}
\end{figure}

\begin{figure}[H]
\centering
\begin{minipage}{.5\textwidth}
  \centering
  \includegraphics[scale=0.4]{img/peces-p-cuarto.png}
  \subcaption{Evolución del número de peces pequeños.}
  \label{fig:peces-p-cuarto}
\end{minipage}%
\begin{minipage}{.5\textwidth}
  \centering
  \includegraphics[scale=0.4]{img/peces-g-cuarto.png}
  \subcaption{Evolución del número de peces grandes.}
  \label{fig:peces-g-cuarto}
\end{minipage}
\caption{Evolución del número de peces de cada población en el tiempo dejando un cuarto de los grandes.}
\end{figure}

Como se puede ver, a grandes rasgos, el comportamiento es bastante similar. Al capturar un número determinado de peces
grandes, la población de peces pequeños crece más o menos dependiendo del número de peces que se hayan capturado. Durante
los siguientes instantes de tiempo, el número de peces pequeños comienza a disminuir y el de grandes a aumentar. El número
de peces de cada población va oscilando hasta llegar a un equilibrio de nuevo, como se puede ver en todas las figuras.

Si estudiamos como evoluciona el número de individuos de cada población de forma conjunta tal y como hicimos anteriormente,
nos encontramos con lo siguiente:

\begin{figure}[H]
\centering
\begin{minipage}{.5\textwidth}
  \centering
  \includegraphics[scale=0.4]{img/peces-pg-mitad.png}
  \subcaption{Dejando la mitad de los grandes.}
  \label{fig:peces-pg-mitad}
\end{minipage}%
\begin{minipage}{.5\textwidth}
  \centering
  \includegraphics[scale=0.4]{img/peces-pg-tercio.png}
  \subcaption{Dejando un tercio de los grandes.}
  \label{fig:peces-pg-tercio}
\end{minipage}
\begin{minipage}{0.5\textwidth}
  \centering
  \includegraphics[scale=0.4]{img/peces-pg-cuarto.png}
  \subcaption{Dejando un cuarto de los grandes.}
  \label{fig:peces-pg-cuarto}
\end{minipage}
\caption{Evolución del número de peces de cada población de forma conjunta.}
\label{fig:peces-pg-captura}
\end{figure}

Como se puede ver en la figura \ref{fig:peces-pg-captura}, aun capturando un número determinado de peces grandes,
se da una de espiral parecida a la que se puede ver en la figura \ref{fig:peces-peq-grand}. La principal diferencia es
la brusca recesión del número de peces grandes, y como consecuente, el incremento del número de peces pequeños, lo cuál
no sucedía anteriormente. Este cambio depende del número de peces grandes que se dejen, ya que dejar un mayor número implica
que la cantidad de peces pequeños crecerá menos.

Como pequeña conlcusión, podemos ver que, a pesar de realizar una captura de un número determinado de peces grandes, si la
población está en equilibrio, volverá a él al cabo de un tiempo, siempre y cuando se deje algún que otro pez grande.

\section{Estudio de políticas de pesca}

Por último, vamos a estudiar una serie de políticas de pesca para ver cuál sería la mejor. El objetivo es intentar
maximizar el número de peces grandes que se capturen en total.

Vamos a suponer que existen tres campañas: una a corto plazo, en la que se hacen capturas cada 30 días (aproximadamente un mes);
una a medio, en la que se hacen capturas cada 90 días (cada 3 meses aproximadamente); y una a largo plazo, en la que se hacen
capturas cada 180 días (aproximadamente cada 6 meses).

Vamos a suponer también que, para cada campaña, se quiere probar a capturar un número distinto de peces grandes para ver qué
efectos tiene en la población. Estas cantidades tienen sentido para los intervalos de tiempo (no se espera que se capturen
el 90\% de los peces en la campaña a corto plazo ya que no tiene mucho sentido capturar tantos). Por tanto, vamos a probar
los siguientes valores:

\begin{itemize}
	\item Para la campaña a \textbf{corto plazo}, vamos a probar a capturar el 10\% de los peces y el 20\%.
	\item Para la campaña a \textbf{medio plazo}, vamos a probar a capturar el 30\% de los peces y el 45\%.
	\item Para la campaña a \textbf{largo plazo}, vamos a probar a capturar el 55\% de los peces y el 75\%.
\end{itemize}

Para realizar la simulación, vamos a simular 20000 días, comenzando con 1500 peces pequeños y 10 grandes.

Una vez realizadas las simulaciones, se han obtenido los siguientes resultados:

\begin{figure}[H]
\centering
\includegraphics[scale=0.43]{img/campanas.png}
\caption{Número de peces grandes capturados para cada campaña.}
\label{fig:campaigns}
\end{figure}

Parece que las campañas que ofrecen unos mejores resultados son aquellas en las que se captura un mayor número de peces,
ya sea a corto, medio o largo plazo. Sin embargo, la que permite maximizar el número de peces capturados es una a corto
plazo en la que se capturan el 20\% de los peces grandes. Si estudiamos las poblaciones de peces, nos encontramos 
con lo siguiente:

\begin{figure}[H]
\centering
\begin{minipage}{.5\textwidth}
  \centering
  \includegraphics[scale=0.4]{img/peces-c-p.png}
  \subcaption{Evolución de los peces pequeños.}
  \label{fig:peces-c-p}
\end{minipage}%
\begin{minipage}{.5\textwidth}
  \centering
  \includegraphics[scale=0.4]{img/peces-c-g.png}
  \subcaption{Evolución de los peces grandes.}
  \label{fig:peces-c-g}
\end{minipage}
\begin{minipage}{0.5\textwidth}
  \centering
  \includegraphics[scale=0.4]{img/peces-c-pg.png}
  \subcaption{Evolución de las dos poblaciones.}
  \label{fig:peces-c-pg}
\end{minipage}
\caption{Evolución del número de peces de cada población según el tiempo y de forma conjunta.}
\label{fig:peces-corto}
\end{figure}

Como se puede ver, las dos poblaciones están en un ``margen de estabilidad''. Esto es, crecen y decrecen dentro de un intervalo
determinado, pero siempre de forma estable y periódica. Por tanto, de aquí podemos conclui que la campaña es relativamente buena,
ya que permite recuperar la población de peces grandes bastante bien y no afecta mucho a la de peces pequeños. Además, es
la que maximiza el número de peces capturados, lo cuál era nuestro objetivo desde el principio.

\end{document}

