\documentclass[11pt,a4paper]{article}
\usepackage[spanish,es-nodecimaldot]{babel}	% Utilizar español
\usepackage[utf8]{inputenc}					% Caracteres UTF-8
\usepackage{graphicx}						% Imagenes
\usepackage[hidelinks]{hyperref}			% Poner enlaces sin marcarlos en rojo
\usepackage{fancyhdr}						% Modificar encabezados y pies de pagina
\usepackage{float}							% Insertar figuras
\usepackage[textwidth=390pt]{geometry}		% Anchura de la pagina
\usepackage[nottoc]{tocbibind}				% Referencias (no incluir num pagina indice en Indice)
\usepackage{enumitem}						% Permitir enumerate con distintos simbolos
\usepackage[T1]{fontenc}					% Usar textsc en sections
\usepackage{amsmath}						% Símbolos matemáticos

% Comando para poner el nombre de la asignatura
\newcommand{\asignatura}{Simulación de Sistemas}
\newcommand{\autor}{Vladislav Nikolov Vasilev}
\newcommand{\titulo}{Problema 1}
\newcommand{\subtitulo}{Modelo de Montecarlo}

% Configuracion de encabezados y pies de pagina
\pagestyle{fancy}
\lhead{\autor{}}
\rhead{\asignatura{}}
\lfoot{Grado en Ingeniería Informática}
\cfoot{}
\rfoot{\thepage}
\renewcommand{\headrulewidth}{0.4pt}		% Linea cabeza de pagina
\renewcommand{\footrulewidth}{0.4pt}		% Linea pie de pagina

\begin{document}
\pagenumbering{gobble}

% Pagina de titulo
\begin{titlepage}

\begin{minipage}{\textwidth}

\centering

\includegraphics[scale=0.5]{img/ugr.png}\\

\textsc{\Large \asignatura{}\\[0.2cm]}
\textsc{GRADO EN INGENIERÍA INFORMÁTICA}\\[1cm]

\noindent\rule[-1ex]{\textwidth}{1pt}\\[1.5ex]
\textsc{{\Huge \titulo\\[0.5ex]}}
\textsc{{\Large \subtitulo\\}}
\noindent\rule[-1ex]{\textwidth}{2pt}\\[3.5ex]

\end{minipage}

\vspace{0.5cm}

\begin{minipage}{\textwidth}

\centering

\textbf{Autor}\\ {\autor{}}\\[2.5ex]
\textbf{Rama}\\ {Computación y Sistemas Inteligentes}\\[2.5ex]
\vspace{0.3cm}

\includegraphics[scale=0.3]{img/etsiit.jpeg}

\vspace{0.7cm}
\textsc{Escuela Técnica Superior de Ingenierías Informática y de Telecomunicación}\\
\vspace{1cm}
\textsc{Curso 2019-2020}
\end{minipage}
\end{titlepage}

\pagenumbering{arabic}
\tableofcontents
\thispagestyle{empty}				% No usar estilo en la pagina de indice

\newpage

\setlength{\parskip}{1em}

\section{\textsc{Descripción del problema}}

Una pequeña fábrica alimenticia se dedica a la producción de caramelos y huevos de Pascua.
Cada año, la fabrica recibe pedidos durante la primera semana de diciembre de huevos de
pascua Pascua de distintas confiterías. La demanda total de huevos varía año tras año, pero
suele seguir una distribución triangular con valor más probable $= 2600$ unidades, menor = $2000$
unidades y mayor $= 3000$ unidades.

Debido a razones estacionales, resulta más barato comprar el chocolate necesario para la
producción de los huevos durante el mes de agosto. Por este motivo, la empresa suele adquirir
una gran cantidad de chocolate durante este mes, y en caso de que cuando reciba los pedidos
sea necesario adquirir más chocolate, se obtiene la cantidad adicional necesaria para satisfacer
la demanda de forma exacta. Si se da el caso de que el chocolate comprado en agosto supera las
necesidades de producción, se dona la cantidad restante a comedores escolares. Se sabe además que:

\begin{itemize}[label=\textbullet]
	\item Cada huevo de Pascua emplea 250 gramos de chocolate.
	\item El precio del chocolate en agosto es de 1 euro por kilo.
	\item El precio del chocolate en diciembre es de $1.5$ euros por kilo.
	\item El precio de venta de los huevos de Pascua es de $0.60$ euros la unidad.
\end{itemize}

Nuestro objetivo es construir un modelo de simulación que ayude a la empresa a determinar
cuántos kilos de chocolate se deberían comprar en el mes de agosto para optimizar el nivel
de las ganancias.

\section{\textsc{Descripción del modelo construido}}

\section{\textsc{Experimentación y resultados}}

\end{document}

